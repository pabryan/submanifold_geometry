\documentclass{article}
\usepackage[utf8]{inputenc}

\title{Submanifold Geometry}
\author{Paul Bryan}

\begin{document}

\maketitle

\section{Isometric Sub-manifold Geometry}

Let $F: (M, \bar{g}) \to (N, \hat{g})$ be an isometric immersion. Let $\bar{\nabla}$ and $\hat{\nabla}$ denote the respective Levi-Civita connections. On the pull-back bundle $F^{\ast} TN$, we have an induced metric $\hat{g}$ and metric-compatible connection $\hat{\nabla}$. The metric $\hat{g}$ splits the  pull-back bundle into orthogonal subspaces,
\[
F^{\ast}TN \simeq TM \oplus \nu M.
\]
Here we abuse notation, but writing $TM$ for $F_{\ast} TM$ since if $F$ is an immersion, $F_{\ast} : TM \to F^{\ast} TN$ is an injective bundle map and hence isomorphic with it's image. Let $\nu \in \Gamma(F^{\ast}TN)$ be a smooth, unit normal vector field along $M$, i.e. $\hat{g}(\nu, \nu) \equiv 1$ and for any $X \in TM$, $\hat{g}(X, \nu) = 0$.

Let $X, Y \in \Gamma(TM) \subset \Gamma(F^{\ast}TN)$, and define the second fundamental form, $\bar{h} \in \Gamma(T^{\ast} M \otimes T^{\ast} M \otimes \nu M)$,
\[
\bar{h}(X, Y) = \pi^{\nu M} \hat{\nabla}_X Y = \hat{\nabla}_X Y - \pi^{TM} \hat{\nabla}_X Y = \hat{\nabla}_X Y - \bar{\nabla}_X Y,
\]
by the Gauss equation: $\bar{\nabla}_X Y = \pi^{TM} \hat{\nabla}_X Y$. We may thus write
\[
\hat{\nabla}_X Y = \bar{\nabla}_X Y + h(X, Y),
\]
with respect to the splitting $F^{\ast}TN = TM \oplus \nu M$.

Also define, $A \in \Gamma(T^{\ast}M \otimes T^{\ast} M)$,
\[
\bar{A}(X, Y) = \hat{g}(\bar{h}(X, Y), \nu) = \bar{g} (\hat{\nabla}_X Y, \nu),
\]
the second fundamental form with respect to $\nu$, so that,
\[
\bar{h}(X, Y) = \bar{A}(X, Y) \nu.
\]
Define the Weingarten map with respect to $\nu$, $W \in \Gamma(T^{\ast}M \otimes \nu M)$ by
\[
W(X) = - \hat{\nabla}_X \nu.
\]

Since $\hat{g}(Y, \nu) = 0$ for any tangent field, we have
\[
0 = \hat{\nabla}_X \hat{g} (Y, \nu) = \hat{g}(\hat{\nabla}_X Y, \nu) + \hat{g} (Y, \hat{\nabla}_X \nu) = \bar{A}(X, Y) - \hat{g}(Y, W(X)).
\]
Thus we find that,
\[
\bar{g} (W(X), Y) = \hat{g} (W(X), Y) = A(X, Y) = A(Y, X) = \hat{g}(X, W(Y)) = \bar{g}(X, W(Y)),
\]
and so $W$ is self adjoint with respect to $\bar{g}$. Observe that we do not furnish $W$ with a bar since it's definition is independent of the metric $\bar{g}$. However, as we have just seen, since $F$ is isometric, $W$ is self-adjoint with respect to $\bar{g}$ through it's relation to the second fundamental form.

\section{Non Isometric Submanifold Geometry}
\end{document}
